\documentclass{ximera}

\title{Section 1}

\begin{document}
\begin{abstract}
Introduction to probability.
\end{abstract}
\maketitle

This section introduces the basic concepts of probability. \vspace{.25cm}

We will consider experiments whose outcomes cannot be predicted with certain. Examples of an experiment include rolling a die or flipping two coins.

The {\it outcome space} $\Omega$ of an experiment is the set of all possible outcomes for the experiment. The outcome space for the experiment of rolling a dice is $\{\,1,2,3,4,5,6\,\}$ and the outcome space for the experiment of flipping two coins is $\{\, \text{Head-Head , Head-Tail, Tail-Head, Tail-Tail} \,\}$.

An {\it event} $A$ is a subset of the sample space: the event of rolling an even number with a dice consist of three outcomes $\{\,2,4,6\,\}$ and the event of flipping two tails with two coins consist of one outcome $\{\, \text{Tail-Tail} \,\}$. 

When the outcome of an experiment is just as likely as another, as in the previous two examples of experiments, the outcomes are said to be {\it equally likely}. \vspace{.25cm}

The following definition is from Wikipedia: 

\href{http://en.wikipedia.org/wiki/Probability}{Probability} is a measure of the likeliness that an event will occur. \vspace{.25cm}

If the number of all outcomes $\Omega$ of an experiment is finite and are equally likely, then the probability of an event $A$ is defined by
\[
\text{Pr}(A) = \frac{ \#(A) }{ \#(\Omega)}.
\]

In section $6.1$ from \href{ http://www.feynmanlectures.caltech.edu/I_06.html#Ch6-S1}{The Feynman Lectures on Physics}, Feynman gives a nice explanation on how to justify the previous formula. If you have time you must read all Chapter $6$. \vspace{.25cm}

Probabilities defined by this formula for equally likely outcomes are fractions between $0$ and $1$. The number $1$ represents certainty and the number $0$ represents impossibility. 
For our examples we have:

\begin{align*}
\text{Pr}&( \text{rolling an even number with a dice} ) = \frac{3}{6}\,,  \\
\text{Pr}&( \text{flipping two tails with two coins} ) = \frac{1}{4}\,.
\end{align*}

\end{document}