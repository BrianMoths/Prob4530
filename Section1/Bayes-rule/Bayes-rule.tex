\documentclass{ximera}

\newcommand{\R}{\mathbb R}

\newcommand{\href}[2]{#2\footnote{\url{#1}}}
\newcommand{\verticalvector}[1]{\begin{bmatrix}#1\end{bmatrix}}
\newcommand{\gt}{>} %% we can turn off input when making a master document

\outcome{learn how to use Bayes' rule.}

\title{Bayes' rule.}

\begin{document}
\begin{abstract}
Bayes' rule
\end{abstract}
\maketitle

{\it For the following problems a suitable tree diagram can help you.}\vspace{.25cm}

{\bf Polya's urn scheme.} An urn contains $4$ white balls and $6$ black balls. A ball is chosen at random, and its color noted. The ball is then replaced, along with $3$ more balls of the same color (so that there are now $13$ balls in the urn). Then another ball is drawn at random from the urn.

\begin{question}
a) Find the chance that the second ball drawn is white.
     \begin{solution}
          The answer is \answer{$2/5$}.
      \end{solution}
\end{question}

\begin{question}
b) Given that the second ball drawn is white, what is the probability that the first ball drawn is black ?
     \begin{solution}
          The answer is \answer{$6/13$}.
      \end{solution}
\end{question}
      
(Think about this generalization) Suppose the original contents of the urn are $w$ white and $b$ blacks balls, and that after a ball is drawn from the urn, it is replaced along with $d$ more balls of the same color. Show that the chance that the second ball drawn is white is $\frac{w}{w+b}$. \vspace{.35cm}

{\bf False diagnostic.} The fraction of persons in a population who have a certain disease is $0.01$. A diagnostic test is available to test for the disease. But for a healthy person the chance of being falsely diagnosed as having the disease is $0.05$, while for someone with the disease the chance of being falsely diagnosed as healthy is $0.2$. Suppose the test is performed on a person selected at random from the population.

\begin{question}
a) What is the probability that the test shows a positive result (meaning the person is diagnoses as diseased, perhaps correctly, perhaps no?)
     \begin{solution}
          The answer is \answer{$0.0575$} .
      \end{solution}
\end{question}

\begin{question}
b) What is the probability that the person selected at random is one who has the disease but is diagnosed healthy?
     \begin{solution}
          The answer is \answer{$0.002$}.
      \end{solution}
\end{question}

\begin{question}
c) What is the probability that the person is correctly diagnosed and is healthy?
     \begin{solution}
          The answer is \answer{$0.9405$}.
      \end{solution}
\end{question}

\begin{question}
d) Suppose the test shows a positive result. What is the probability that the person tested actually has the disease?
     \begin{solution}
          The answer is \answer{$16/115$}.
      \end{solution}
\end{question}

\end{document}