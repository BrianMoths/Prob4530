\documentclass{ximera}

%\newcommand{\R}{\mathbb R}

\newcommand{\href}[2]{#2\footnote{\url{#1}}}
\newcommand{\verticalvector}[1]{\begin{bmatrix}#1\end{bmatrix}}
\newcommand{\gt}{>} %% we can turn off input when making a master document

\outcome{To solve problems related to Poisson distribution.}

\title{Poisson distribution}

\begin{document}
\begin{abstract}
A few more exercises about Poisson random variables
\end{abstract}
\maketitle

Let $X$ be a Poisson random variable with parameter $2$. Calculate 

\begin{question}
a) $E(3X+5)$
     \begin{solution}
          The answer is \answer{$11$}.
     \end{solution}
\end{question}

\begin{question}
b) $\text{Var}(3X+5)$
     \begin{solution}
          The answer is \answer{$18$}.
     \end{solution}
\end{question}

\begin{question}
c) $E\bigg( \frac{1}{1+X} \bigg)$
     \begin{solution}
          The answer is \answer{$( 1-e^(-2) )/2$}.
     \end{solution}
\end{question}

A book has $250$ pages. The number of mistakes on each page is a Poisson random variable with mean $0.02$, and is independent of the number of mistakes on all other pages.

\begin{question}
a) What is the expected number of pages with no mistakes ?
     \begin{solution}
          The answer is \answer{$245.05$} (round to two decimals).
     \end{solution}
\end{question}

\begin{question}
b) A person proofreading the book finds a given mistake with probability $0.9$. What is the expected number of pages where this person will find a mistake ? 
     \begin{solution}
          The answer is \answer{$4.46$} (round to two decimals).
     \end{solution}
\end{question}

\begin{question}
b) What, approximately, is the probability that the book has two or more pages with mistakes ? 
     \begin{solution}
          The answer is \answer{$0.96$} (round to two decimals).
     \end{solution}
\end{question}

\end{document}
