\documentclass{ximera}

%\newcommand{\R}{\mathbb R}

\newcommand{\href}[2]{#2\footnote{\url{#1}}}
\newcommand{\verticalvector}[1]{\begin{bmatrix}#1\end{bmatrix}}
\newcommand{\gt}{>} %% we can turn off input when making a master document

\outcome{learn how to calculate moment and probability generating functions.}

\title{Generating functions.}

\begin{document}
\begin{abstract}
Calculations
\end{abstract}
\maketitle

\begin{question} 
Let $X$ and $Y$ be independent random variables with moment generating functions $f(t)$ and $g(t)$, respectively. Let $Z=X+Y$. Determine $M_{Z}(t)$, the moment generating function of $Z$, in terms of $f(t)$ and $g(t)$.
     \begin{hint}
          $E(e^{tZ}) = E(e^{tX}e^{tY})$
     \end{hint}
     \begin{hint}
          The answer must be expressed in terms of $f(t)$ and $g(t)$.
     \end{hint}
     \begin{solution}
          The moment generating function of $Z$ is \answer{$f(t)g(t)$}.
     \end{solution}
\end{question}

\begin{question}
Let $X$ and $Y$ be independent random variables that are both geometric distributions with likelihood of success equal to $p$. Using the previous problem, determine the moment generating function of $Z=X+Y$.
     \begin{hint}
          The answer must be an algebraic expression in terms of $p$, $e$ and $t$.
     \end{hint}
     \begin{solution}
          The moment generating function of $Z$ is \answer{$(  (  p e^t ) / (   1 - e^t (1-p)  )  )^2 $}.
     \end{solution}
\end{question}

\begin{question}
Let $X$ have a Poisson distribution with expectation $m$. Determine the probability generating function $G(z)$ of $X$.
     \begin{hint}
          The answer must be an algebraic expression in terms of $m$, $e$ and $z$.
     \end{hint}
     \begin{solution}
          The probability generating function of $X$ is \answer{$e^(m*(z-1))$}.
     \end{solution}
\end{question}

\end{document}
