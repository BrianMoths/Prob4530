\documentclass{ximera}

%\newcommand{\R}{\mathbb R}

\newcommand{\href}[2]{#2\footnote{\url{#1}}}
\newcommand{\verticalvector}[1]{\begin{bmatrix}#1\end{bmatrix}}
\newcommand{\gt}{>} %% we can turn off input when making a master document

\outcome{to be able to recognize the binomial distribution.}

\title{Repeated trials.}

\begin{document}
\begin{abstract}
mathematical model for repeated trials.
\end{abstract}
\maketitle

Given that there were $13$ heads in $25$ independent coin tosses, calculate

\begin{question}
The chance that the first toss landed head
     \begin{solution}
          The answer is \answer{$13/25$}.
     \end{solution}
\end{question}

\begin{question}
The chance that the first two tosses landed heads.
     \begin{solution}
          The answer is \answer{$13/50$}.
     \end{solution}
\end{question}

\begin{question}
The chance that at least two of the first five tosses landed heads.
     \begin{solution}
          The answer is \answer{$1391/1610$}.
     \end{solution}
\end{question}

\begin{question}
You roll a die, and I roll a die. You win if the number showing on your die is strictly greater than the one on mine. If we play this game five times, what is the chance that you  win at least three times ?
      \begin{solution}
          The answer is \answer{$14375/41472$}.
     \end{solution}
\end{question}

$68\,\%$ of the people in a certain population are adults. A random sample of size $20$ will be drawn, with replacement, from this population. 

\begin{question}
What is the most likely number of adults in the sample ?
     \begin{solution}
          The answer is \answer{14}.
     \end{solution}
\end{question}

\begin{question}
What is the chance of getting exactly this many adults ?
     \begin{solution}
          The answer is \answer{18.81} $\,\%$. (round to two decimal places)
     \end{solution}
\end{question}

\end{document}
