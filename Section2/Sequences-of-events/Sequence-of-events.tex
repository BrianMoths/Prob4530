\documentclass{ximera}

%\newcommand{\R}{\mathbb R}

\newcommand{\href}[2]{#2\footnote{\url{#1}}}
\newcommand{\verticalvector}[1]{\begin{bmatrix}#1\end{bmatrix}}
\newcommand{\gt}{>} %% we can turn off input when making a master document

\outcome{generalization of conditional probability and independence.}

\title{Sequence of events / Independence.}

\begin{document}
\begin{abstract}
probabilities determined by a sequences of events or outcomes.
\end{abstract}
\maketitle

\begin{question} 
     {\bf The birthday problem.} Suppose you are one of $n$ students in a class. What is the smallest value of $n$ such that $P(\text{at least 2 birthdays coincide})\geq \tfrac{1}{2}$?
     \begin{solution}
          The smallest value of $n$ is \answer{$23$}.
     \end{solution}
\end{question}

\begin{question}
There are twelve signs of the zodiac. How many people must be present for there to be at least a $50\,\%$ chance that two or more of them were born under the same sign ?
     \begin{solution}
     The answer is \answer{$5$}.
     \end{solution}
\end{question}

Suppose that $A$, $B$, and $C$ are pairwise independent.  

$ P(A) = 0.3 $

$ P(B) = 0.4 $

$ P(A\cap B\cap C) = 0$

$ P(A\cup B\cup C) = 0.73$

\begin{question}
Use the information above to determine $P(C)$.
     \begin{hint}
          inclusion - exclusion formula.
     \end{hint}
     \begin{solution}
     $P(C) = $ \answer{$0.5$}. 
     \end{solution}
\end{question}

(Think about the following questions)  Are $A$, $B$, and $C$ independent ? Why ?

\begin{question}
A couple has two children.  Determine the probability that the couple has two girls given that they have one girl that was born between 5am and 8am.     
     \begin{solution}
           The answer is \answer{$15/31$}.
     \end{solution}
\end{question}

\end{document}
