\documentclass{ximera}

\title{Section 5}

\begin{document}
\begin{abstract}
Joint distributions.
\end{abstract}
\maketitle

There are many situations which involve presence of several random variables and we are interested in their joint behavior. \vspace{.25cm}

Suppose that $X$ and $Y$ are two random variables defined on the same outcome space $\Omega$. The joint cumulative distribution function of X and Y is the function
\begin{equation*}
F_{XY}(x,y)=\text{Pr}(X\leq x,Y\leq y)=\text{Pr}(\{ s\in\Omega : X(s)\leq x \text{ and } Y(s)\leq y \}).
\end{equation*}

Two random variable $X$ and $Y$ are said to be joint continuous if there exists a function $f_{XY}(x,y)\geq 0$ with the property that for every subset $C$ of $\mathbb{R}^{2}$ we have
\begin{equation*}
\text{Pr}((X,Y)\in C)=\int\int_{(x,y)\in C} f_{XY}(x,y)dxdy.
\end{equation*}
The function $f_{XY}(x,y)$ is called the joint probability density function of $X$ and $Y$.

\end{document}
